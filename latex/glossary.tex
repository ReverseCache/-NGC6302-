\documentclass{article}

\usepackage{comment, multicol}
\usepackage{hyperref}

\usepackage{calc,pict2e,picture}
\usepackage{textgreek,textcomp,gensymb,stix}

\setcounter{secnumdepth}{2}

\title{Glossary}
\author{made with \LaTeX.js and JavaScript}
\date{Compiled by Amruth and Ryan on 2021 June}


\begin{document}

\maketitle


\begin{abstract}
This document aims to illustrate the features of fundamental indicators  while at the same time serving as a gentle introduction to fundamental analysis.
Furthermore, formulae are provided as to further help with demystifying the indicators. 
\end{abstract}

\section{Terms}
\begin{itemize}
    
    \item \texttt{ Current Ratio}: \normalsize{Section~\ref{sec:currentRatio}}
    \item \texttt{ Longterm Debt to Capital}: \normalsize{Section~\ref{sec:ldtc}}
    \item \texttt{ Debt To Equity Ratio}: \normalsize{Section~\ref{sec:dter}}
    \item \texttt{ Gross Margin}: \normalsize{Section~\ref{sec:GM}}
    \item \texttt{ Operating Margin}: \normalsize{Section~\ref{sec:OM}}
 	\item \texttt{ Ebit Margin}: \normalsize{Section~\ref{sec:em}}
    \item \texttt{ Ebita Margin}: \normalsize{Section~\ref{sec:em2}}
    \item \texttt{ Pretax Profit Margin}: \normalsize{Section~\ref{sec:ppm}}
    \item \texttt{ Net Profit Margin}: \normalsize{Section~\ref{sec:npm}}
    \item \texttt{ Asset Turnover}: \normalsize{Section~\ref{sec:at}}
    \item \texttt{ Inventory Turnover}: \normalsize{Section~\ref{sec:it}}
    \item \texttt{ Receivables Turnover Ratio}: \normalsize{Section~\ref{sec:Rtr}}
	\item \texttt{ Day Sales In Receivables}: \normalsize{Section~\ref{sec:dsir}}
    \item \texttt{ Return on Equity}: \normalsize{Section~\ref{sec:roe}}
	\item \texttt{ Return on tangible equity}: \normalsize{Section~\ref{sec:rote}}
	\item \texttt{ Return on investment}: \normalsize{Section~\ref{sec:roi}}
	\item \texttt{ Return on Assets}: \normalsize{Section~\ref{sec:roa}}
    \item \texttt{ Book Value Per Share}: \normalsize{Section~\ref{sec:bvps}}
	\item \texttt{ Operating Cash Flow Per Share}: \normalsize{Section~\ref{sec:ocfps}}
    \item \texttt{ Free Cash Flow Per Share}: \normalsize{Section~\ref{sec:fcfps}}
\end{itemize}

\section{Current Ratio}
\label{sec:currentRatio}

The current ratio is a liquidity ratio that measures a company's ability to pay short-term obligations or those due within one year. It tells investors and analysts how a company can maximize the current assets on its balance sheet to satisfy its current debt and other payables. 
\bigskip

Formula:
$$
\textbf{Current Ratio} =  \frac{\textbf{Current Assets}}{\textbf{Current Liabilites}}
$$


\bigskip
\section{Longterm Debt to Capital}
\label{sec:ldtc}
A Long Term Debt to Capitalization Ratio is the ratio that shows the financial leverage of the firm. This ratio is calculated by dividing the long term debt with the total capital available of a company. The total capital of the company includes the long term debt and the stock of the company. This ratio allows the investors to identify the amount of control utilized by a company and compare it to other companies to analyze the total risk experience of that particular company.
\bigskip

Formula:
$$
\textbf{Longterm Debt to Capital} =  \frac{\textbf{Long term debt}}{\textbf{Long term debt + Preferred Stock + Common Stock}}
$$

\section{Debt To Equity Ratio}
\label{sec:dter}
The D/E ratio is an important metric used in corporate finance. It is a measure of the degree to which a company is financing its operations through debt versus wholly owned funds. More specifically, it reflects the ability of shareholder equity to cover all outstanding debts in the event of a business downturn. The debt-to-equity ratio is a particular type of gearing ratio. 

\bigskip
Formula:
$$
\textbf{Debt to Equity} =  \frac{\textbf{Total Liabilities}}{\textbf{Total Shareholder Equity}}
$$

\section{Gross Margin}
\label{sec:GM}
The Gross Profit Margin shows the income a company has left over after paying off all direct expenses related to the manufacturing of a product or providing a service. Gross Margin is an indicator of whether a company is running an efficient operation and if its sales are good enough.

\bigskip
Formula:
$$
\textbf{Debt to Equity} =  \frac{\textbf{(Total Revenue – Cost of Goods Sold)*100}}{\textbf{Total Revenue}}
$$

\section{Operating Margin}
\label{sec:OM}
Operating margin measures how much profit a company makes on a dollar of sales after paying for variable costs of production, such as wages and raw materials, but before paying interest or tax. It is calculated by dividing a company’s operating income by its net sales. Higher ratios are generally better, illustrating the company is efficient in its operations and is good at turning sales into profits.

\bigskip
Formula:
$$
\textbf{Operating Margin} =  \frac{\textbf{(Operating Profit)*100}}{\textbf{Net Sales}}
$$

\section{Ebit Margin}
\label{sec:em}
An EBIT Margin is the operating earnings over operating sales. This margin allows investors to understand true business costs of running a company, because parts of a company's property, plant, and equipment will eventually need to be replaced as they get used, broken down, decayed, etc.
\par
\bigskip
Lower EBIT Margins indicate lower profitability from a company. When comparing against its competitors, investors can determine if lower EBIT margins are due to the competitive landscape (where all companies are having lower margins) or a issue just within the company (where the company is facing lower sales and higher costs).
\bigskip

Formula:
$$
\textbf{EBIT Margin} =  \frac{\textbf{Last four quarters of operating earnings}}{\textbf{Last four quarters of sales}}
$$

\section{Ebita Margin}
\label{sec:em2}
The EBITDA margin is a measure of a company's operating profit as a percentage of its revenue. The acronym EBITDA stands for earnings before interest, taxes, depreciation, and amortization. Knowing the EBITDA margin allows for a comparison of one company's real performance to others in its industry. 
\par
\bigskip

Formula:
$$
\textbf{EBITA Margin} =  \frac{\textbf{Earnings before interest and tax + Depreciation + Amortization}}{\textbf{total revenue }}
$$

\section{Pretax Profit Margin}
\label{sec:ppm}
The pretax profit margin is a financial accounting tool used to measure the operating efficiency of a company. It is a ratio that tells us the percentage of sales that has turned into profits or, in other words, how many cents of profit the business has generated for each dollar of sale before deducting taxes. The pretax profit margin is widely used to compare the profitability of businesses within the same industry. 
\par
\bigskip

Formula:
$$
\textbf{Pretax Profit Margin} =  \frac{\textbf{Earnings before interest}}{\textbf{Sales }}
$$

\section{Net Profit Margin}
\label{sec:npm}
The net profit margin, or simply net margin, measures how much net income or profit is generated as a percentage of revenue. It is the ratio of net profits to revenues for a company or business segment. Net profit margin is typically expressed as a percentage but can also be represented in decimal form. The net profit margin illustrates how much of each dollar in revenue collected by a company translates into profit. 
\par
\bigskip

Formula:
$$
\textbf{Net Profit Margin} =  \frac{\textbf{(Net Income) * 100 }}{\textbf{Revenue }}
$$

\section{Asset Turnover}
\label{sec:at}
 The asset turnover ratio measures the value of a company's sales or revenues relative to the value of its assets. The asset turnover ratio can be used as an indicator of the efficiency with which a company is using its assets to generate revenue.
\par
\bigskip
The higher the asset turnover ratio, the more efficient a company is at generating revenue from its assets. Conversely, if a company has a low asset turnover ratio, it indicates it is not efficiently using its assets to generate sales. 
\par
\bigskip

Formula:
$$
\textbf{Asset Turnover} =  \frac{\textbf{Total Sales}}{\frac{\textbf{Beginning Assets + Ending Asset}}{\textbf{2 }}}
$$

\section{Inventory Turnover}
\label{sec:it}
Inventory turnover is a financial ratio showing how many times a company has sold and replaced inventory during a given period. A company can then divide the days in the period by the inventory turnover formula to calculate the days it takes to sell the inventory on hand. 
\par
\bigskip

Formula:
$$
\textbf{Inventory Turnover} =  \frac{\textbf{Cost of goods sold
}}{\textbf{ Average Value of Inventory}}
$$

\section{Receivables Turnover Ratio}
\label{sec:Rtr}
The receivables turnover ratio is an accounting measure used to quantify a company's effectiveness in collecting its accounts receivable, or the money owed by customers or clients. This ratio measures how well a company uses and manages the credit it extends to customers and how quickly that short-term debt is collected or is paid. A firm that is efficient at collecting on its payments due will have a higher accounts receivable turnover ratio. 
\par
\bigskip
Formula:
$$
\textbf{Accounts Receivable Turnover} =  \frac{\textbf{Net Credit Sales
}}{\textbf{Average Accounts Receivable}}
$$

\section{Day Sales In Receivables}
\label{sec:dsir}
The days' sales in accounts receivable ratio (also known as the average collection period) tells you the number of days it took on average to collect the company's accounts receivable during the past year. 
\par
\bigskip
Formula:
$$
\textbf{Day Sales In Receivables} =  \frac{\textbf{Number of days in the year}}{\textbf{Accounts Receivable Turnover
}}
$$

\section{Return on Equity}
\label{sec:roe}
Return on equity (ROE) is a measure of financial performance calculated by dividing net income by shareholders' equity. Because shareholders' equity is equal to a company’s assets minus its debt, ROE is considered the return on net assets. ROE is considered a measure of a corporation's profitability in relation to stockholders’ equity. 
\par
\bigskip
Formula:
$$
\textbf{Return on Equity} =  \frac{\textbf{Net Income}}{\textbf{Average Shareholders’ Equity
}}
$$

\section{Return on tangible equity}
\label{sec:rote}
Return on tangible equity or ROTE is the net profit (after interest and tax) as a percentage of the (average) tangible equity or shareholders' funds. Tangible equity is equity or net assets less intangible assets such as goodwill.
\par
\bigskip
Formula:
$$
\textbf{Return on tangible equity} =  \frac{\textbf{Net earnings applicable to common shareholders }}{\textbf{Average monthly tangible common shareholders' equity}}
$$

\section{Return on investment}
\label{sec:roi}
Return on investment (ROI) is the ratio of a profit or loss made in a fiscal year expressed in terms of an investment. It is expressed in terms of a percentage of increase or decrease in the value of the investment during the year in question. For example, if you invested 100 in a share of stock and its value rises to 110 by the end of the fiscal year, the return on the investment is a healthy 10 percent, assuming no dividends were paid. 
\par
\bigskip
Formula:
$$
\textbf{Return on investment} =  \frac{\textbf{Final value, including dividends and interest − Initial value
}}{\textbf{Initial value}}
$$

\section{Return on Assets}
\label{sec:roa}
Return on assets (ROA) is an indicator of how profitable a company is relative to its total assets. ROA gives a manager, investor, or analyst an idea as to how efficient a company's management is at using its assets to generate earnings.
\par
\bigskip
Formula:
$$
\textbf{Operating Cash Flow Per Share} =  \frac{\textbf{Net Income}}{\textbf{Total Assets}}
$$

\section{Book Value Per Share}
\label{sec:bvps}
Book value per share (BVPS) is the ratio of equity available to common shareholders divided by the number of outstanding shares. This figure represents the minimum value of a company's equity and measures the book value of a firm on a per-share basis. 
\par
\bigskip
Formula:
$$
\textbf{Operating Cash Flow Per Share} =  \frac{\textbf{Total Equity − Preferred Equity
}}{\textbf{Number of Shares Outstanding}}
$$

\section{Operating Cash Flow Per Share}
\label{sec:ocfps}
Operating Cash Flow Per Share refers to the amount of cash a company generates from the revenues it brings in, excluding costs associated with long-term capital investment. It is similar to operating profit but excluding non-cash items and accruals. This is measured on a TTM basis. using Diluted Shares Outstanding.
\par
\bigskip
Formula:
$$
\textbf{Operating Cash Flow Per Share} =  \frac{\textbf{Net Income + Non Cash Expense - Increase in Working Capital
}}{\textbf{Number of Shares Outstanding}}
$$

\section{Free Cash Flow Per Share}
\label{sec:fcfps}
Free cash flow per share (FCF) is a measure of a company's financial flexibility that is determined by dividing free cash flow by the total number of shares outstanding. This measure serves as a proxy for measuring changes in earnings per share. 
\par
\bigskip
Formula:
$$
\textbf{Free Cash Flow per Share} =  \frac{\textbf{Free Cash Flow
}}{\textbf{Number of Shares Outstanding}}
$$

\appendix

\section{Information Sources}

\begin{itemize}
    
    \item \texttt{ https://www.investopedia.com}
    \item \texttt{ https://ycharts.com}
    \item \texttt{ https://corporatefinanceinstitute.com}
	\item \texttt{ https://www.accountingcoach.com}
    \item \texttt{ https://www.readyratios.com}
    
\end{itemize}
\end{document}
